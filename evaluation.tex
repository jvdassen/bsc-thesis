\chapter{Evaluation}
The ecosystem of the Bazo cryptocurrency can not be considered completed to a degree where a test run would be possible, at the time of writing. Reasons for this are that the development of the interfaces of the existing infrastructure was not completed. Since these interfaces were specifically designed for the Bazo Wallet during the course of this thesis, the Wallet itself can be considered complete. This chapter assesses the developed Wallet and explaines limitations to the system.
\section{Prototype Evaluation}
This section evaluates the developed P2P mobile payment application against key characteristics for such a solution. Key characteristics of a mobile payment solution can be:
\begin{enumerate}
\item \textbf{Speed}
The speed of a payment is characterized as the time passed between initiation and clearance of the transaction.
\item \textbf{Payer control}
This refers to the degree as where a user can select the most advantageous terms for his payment.
\item \textbf{Security}
The degree of risk of having funds manipulated or stolen by engaging in the payment system is considered Security.
%inverse?%
\item \textbf{Universality}
This key characteristic deals with the importance of a payment system being accepted by a large user base to make the payment solution attractive.
\end{enumerate}

Considering speed, there are multiple issues for assessing the Speed of the Bazo Wallet. First, it is hard to determine where the start of the transaction process begins. It is assumed that the transaction process starts when the user has already all the necessary information such as target address and transaction value available. It is difficult to assess the speed of the Wallet since the speed of the payment depends on other systems. A high amount of unconfirmed transactions, a lower fee selected by the user and the configuration of the block interval can impact the payment speed. If we would assume these values were chosen optimally, the Bazo Wallet contributes little to no time to the overall time required for the payment. Obtaining transaction data, signing and submitting it from and to a remote Bazo client, took less than 500ms in tests. Using a local client, such as the Android version should require even less time, since most of the time is lost connecting to remote systems.

Payer control with the Bazo currency has a mixed image.
The user has the control to create transactions at terms he prefers, for example, the user can set the fee he is ready to spend for the transaction. However, the clearance of the transaction depends very much on the network. That is the current amount of unconfirmed transactions, the rate at which transactions are validated and the fee the user has set. The user has little to no influence on these conditions.

Security as in the definiton above, should be given for all transactions signed by the user. Section \ref{limitations} outlines cases, where the user can be tricked to reveal his public key. From a technical point of view, the Wallet can be considered secure, as that it is not possible to steal a private key or manipulate transactions which would result in the loss of funds for the user.

Universality is not a strength of the Bazo currency. Since the Wallet is not compatible with other payment systems or applications only users in the system can exchange funds. Since Bazo is a newly created cryptocurrency, there is no user base and users would have to be convinced to join the system. Given that the target users should already be in the bonus program, the usage of the Bazo currency could be offered an incentive. Since the mechanisms in the Wallet for communicating with the currency are very similar to the mechanisms in other currencies such as Ripple, it would technically be feasible to reuse the Wallet for other currencies and map the bonus points to said currency \cite{ripplelib}, thus leveraging an existing user base.

\section{Limitations}\label{limitations}
Since with blockchain based cryptocurrencies it is a goal to be independent on trusting a single authority, the Bazo Wallet should be evaluated with respect to this key factor.
%???%
Since, the application is designed as a signing-only client, there needs to exist a certain amount of trust between the user of such a Wallet and the server he relies on \cite{bitcoinclients}.
It is known that although there is some space for attacks, such as tampering with account balances when queried, it is not possible for a server to modify transactions and therefore steal funds. This is ensured by having all transaction signing implemented in the browser. Since sharing transaction data is not performed over the same server, but rather on a P2P basis, it is also hard to trick the user into targeting the wrong address. Further, the user has the possibility to set a custom URI, which is used for all transactions. That way, one could deploy the light client to an https enabled server and use this node as a backend. This is an approach that is frequently employed when dealing with signing-only clients \cite{bitcoinclients}. It is also technically possible to run the light client locally on an android phone, by running the ARM compiled binaries of the light client. This was tested and proved to work as with any other operating system or platform.
Another risk is introduced with the unified data model of transaction data, since this points to the URI of the Bazo Wallet. One could trick a user into using a web application that looks like the Bazo Wallet, but has the only purpose of stealing the private key, in case that the user does not realize that the URI does not belong to the actual Wallet.
One solution how this weakness can be handled is to install the progressive web application to the home screen. On Android platforms, this will associate the installed application with the URI of the origin. Therefore the application will only open if the host of the URI matches the origin of the installed web page.
\section{Future Work}
During the development of the Bazo Wallet, limitations with respect to the current implementation of applications in the Bazo ecosystem were observed. This chapter summarizes the issues and possible solutions.

Chapter \ref{requirementsanalysis} contained an analysis of the application that was to be developed. Due to the elicited requirements and the design guidelines the application was developed as a signing-only client. Considering only the resources available to such an application, it would be possible to develop a headers-only client for a blockchain based currency. For the Bazo currency, this was not applicable for two reasons. First, a web application can not participate in the peer-to-peer network over a TCP connection. Second, JavaScript can not be used to validate blocks, since it is lacking support for low-level data types. The first of these issues could be solved by implementing a web service that allows a web application to download block headers over an https or wss connection. This web service could be in the form of a standalone application or integrated into the RESTful API of the Bazo Light client. The second issue would have to be solved by changing the way that transactions are signed and verified. The necessary changes would require a change of the protocol. This would imply a fork of the blockchain for a productive system.

Chapter \ref{bazomobile} proved that it is technically possible to run the bazo client and even the bazo miner application on a mobile device running Android. This was done with the intention to allow the PWA to communicate with a trustable Bazo node. The necessity of this solution lies in the weakness of signing-only clients which rely on a web service and thus on a third-party. To allow a user to use the compiled binaries in a comfortable manner the following issues would have to be evaluated and solved:
\begin{itemize}
\item Android wrapper application.
For the user
\end{itemize}
\begin{bclogo}[logo=\bcattention, couleurBarre=red, noborder=true, 
               couleur=LightSalmon]{Important!}
This section is incomplete.
\end{bclogo}