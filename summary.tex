\chapter{Summary and Conclusions}

The overarching goal of this thesis was the enrichment of the Bazo ecosystem to a point, where the bespoke cryptocurrency can be used in a sandboxed environment for mobile payments. This was achieved by implementing a PWA-based Mobile Wallet with the respective interfaces to the system. Through this application, the financial service provider can take the first steps at evaluating the payment system and comparing it to traditional payment systems. Due the strongly web-focused approach of the implementation, the current state of different Web API's was explored, tested and documented. With the development of a PoC, an architecture, where communication between native and web application is possible, was assessed. This revealed strengths and weaknesses of a web-based approach to the development of a Mobile Wallet. Another PoC lied in porting the existing applications to the Android operating system. With this, the technical feasibility of a completely trustless, web-based Wallet was demonstrated; a key factor that other popular Wallets are not capable of.

The introduction of the thesis lied in a motivation and description of work. In the subsequent chapter, related technologies, activities and the elicited requirements were explained to introduce the reader to the context. The design of the envisioned solution was then explained for the aforementioned requirements. The implementation of this design was then detailed in the fourth chapter, where the development of the required functionalities was presented. The Evaluation chapter highlighted strengths and weaknesses of the developed system and outlined open questions that would ideally become the subject of further research.

The developed Bazo Wallet can be used as a fully functional signing only client for the cryptocurrency. Further, the application provides different features, so that it can be used as a mobile payment system. 