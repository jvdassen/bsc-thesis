\chapter*{Abstract}
\addcontentsline{toc}{chapter}{Abstract}

\selectlanguage{english}
Ein Finanzdienstleister aus Zürich unterhält ein Bonusprogramm um die Verwendung von Kredit- und Debitkarten zu belohnen. Der traditionelle Prozess wie Bonuspunkte gegen Produkte getauscht werden können, bedeutete, dass Verträge zu Partnerfirmen unterhalten werden müssen. Dieser Prozess führte dazu, dass das Programm bei Kunden kaum bekannt wurde. Aus diesem Grund wurde an der Universität Zürich die Bazo Krypto-Währung entwickelt, welche nur eingeladenen Teilnehmern offen steht. Um den Finanzdienstleister die Technologie evaluieren zu lassen, wurden weitere Entwicklungsaufwände betrieben um andere übliche Applikationen für Kryptowährungen bereitzustellen. Der Fokus dieser Arbeit liegt auf der Planung, Entwicklung und Evaluation einer Zahlungs-Applikation für die Bazo Währung. Neben dieser Entwicklung befasst sich diese Arbeit auch mit der Erfassung des aktuellen Stands von Web APIs, die normalerweise nur auf nativen Applikationen zur Verfügung stehen. Weiter wurde untersucht wie native APIs einer Web Applikation zur Verfügung gestellt werden können. Konkret wurde dazu ein Proof-of-Concept erstellt, womit native Funktionen einer Web Applikation bereit gestellt werden. Weiter wurden Schnittstellen zu existierenden Systemen, wie dem Bonusprogramm und der Kryptowährung, geplant und entwickelt. Ausserdem wurde im Rahmen der Arbeit evaluiert wie die existierende Client Applikation auf das mobile Betriebssystem Android portiert werden kann.
\selectlanguage{english}

A financial service provider in Zurich maintains a bonus program to incentivize credit and debit card usage. The process of exchanging bonus points against goods implied efforts to maintain contracts with partners and led to low usage of the program by customers. This led to the development of the Bazo cryptocurrency at the University of Zurich, which is an invite-only, blockchain-based currency. In order to let the financial service provider evaluate the technology, further development efforts were taken to enrich the ecosystem of the currency. The scope of this thesis is the design, development and evaluation of a web-based payment application for the Bazo cryptocurrency. Besides the development of the application, this thesis explores related topics, such as a comparison of Web APIs with native APIs by developing a Proof of Concept for the Android operating system. Further, the development efforts for the required interfaces to the currency and to the companys existing infrastructure are described. Finally, a mobile port of the existing client application for the Bazo currency targeted to the Android operating system was created and evaluated.

