\chapter{Introduction}
%% Importance etc of Blockchain based currencies

A financial service provider based in Zurich developed a bonus program that rewards customers when using their credit and debit cards. With every completed purchase, customers collect virtual points. Several industry partners are participants of this bonus program. The collected virtual points can be exchanged for gift cards from partners who are listed on a Web site or an app. The way that these services can be used in the traditional system puts the company into a central role, thus requiring administrative efforts to enable both customers and partners to exchange virtual points. In order to decentralize the financial service provider's position, a blockchain based currency, Bazo (Esperanto: base; foundation), was developed at the University of Zurich. The intention of the currency is to map virtual points to coins in the crypto currency \cite{lisg}.
The focus of this thesis lies on the implementation and evaluation of a Mobile Wallet for said currency. Requirements and implementation details are gathered during the project in an explorative way in partnership with the company.
\section{Motivation}
Traditional payment systems require a centralized institution in order to maintain a currency and support operations such as issueing, transferring and determining the state of currency units. Requiring the issuer for each operation that can be performed on the virtual currency further prohibits the direct trading of virtual points between users.
In the case of the partner company this resulted in a limited awareness and usage of the incentivization program by customers as well as partners.
With the new cryptocurrency developed at the University of Zurich, which is tailored to the requirements of a financial service provider, the company can maintain control over the currency since it remains in an issuer position. In order to let end users such as customers and partners interact with the currency as well as making the program more transparent, further development efforts are taken. With the completion of said efforts, a first complete prototype of a currency ecosystem should be available for evaluation. In doing so, the financial service provider can evaluate the new solution with real costumers and gain insight on how blockchain technologies can be used to improve payment processes. These development efforts include the development of a client application to let users trade Bazo coins as well as to improve the payment process, since mobile system's capabilites for sharing transaction data can be used.
%mobile sysstem's%
\section{Description of Work}
This thesis covers the design, implementation, and evaluation of a mobile wallet for the Bazo cryptocurrency. In order to run the client as trustless as possible, a light client implementation needs to be integrated to the Bazo wallet. This involves the integration and development of interfaces with other applications in the cryptocurrency to enable operations on the blockchain.
The development should leverage existing resources from similar projects where a Mobile Wallet was implemented.
It is required to find a way to transmit payment information with native elements such as NFC and Bluetooth. However, a fallback solution needs to be implemented to allow a broad range of users to participate in the program.
In order to support the complete use case for customers and partners, it is required to integrate existing web services from the service provider. This is intended to support the complete payment process for proximity-based transfers.
The application needs to be tested in a pilot project where the Bazo cryptocurrency is used for payments in a sandboxed environment.

\section{Outline}
The main section of the report is structered as follows:
In chapter 2 an introduction to leveraged and related technologies, as well as the traditional and envisioned transaction process is given. The chapter further contains a documentation of elicited and derived requirements and an analysis thereof. Further, related work, as well as other activities in the project are introduced. Chapter 3 focuses on the design of a Mobile Wallet that satisfies the aforementioned requirements. The concrete implementation of the design is explained in chapter 4. In chapter 5 the prototype and the findings from developing the Wallet are evaluated and compared.
Chapter 6 summarizes and concludes the report.